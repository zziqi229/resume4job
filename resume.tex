%-------------------------
% Resume in Latex
% Author : Sourabh Bajaj
% Website: https://github.com/sb2nov/resume
% License : MIT
%------------------------

\documentclass[letterpaper,11pt]{article}

\usepackage{latexsym}
\usepackage[empty]{fullpage}
\usepackage{titlesec}
\usepackage{marvosym}
\usepackage[usenames,dvipsnames]{color}
\usepackage{verbatim}
\usepackage{enumitem}
\usepackage[hidelinks]{hyperref}
\usepackage{fancyhdr}
\usepackage[UTF8]{ctex}


\pagestyle{fancy}
\fancyhf{} % clear all header and footer fields
\fancyfoot{}
\renewcommand{\headrulewidth}{0pt}
\renewcommand{\footrulewidth}{0pt}

% Adjust margins
\addtolength{\oddsidemargin}{-0.375in}
\addtolength{\evensidemargin}{-0.375in}
\addtolength{\textwidth}{1in}
\addtolength{\topmargin}{-.5in}
\addtolength{\textheight}{1.0in}

\urlstyle{same}

\raggedbottom
\raggedright
\setlength{\tabcolsep}{0in}

% Sections formatting
\titleformat{\section}{
  \vspace{-4pt}\scshape\raggedright\large
}{}{0em}{}[\color{black}\titlerule \vspace{-5pt}]

%-------------------------
% Custom commands
\newcommand{\resumeItem}[2]{
  \item\small{
    \textbf{#1}{ #2 \vspace{-2pt}}
  }
}

\newcommand{\resumeSubheading}[4]{
  \vspace{-1pt}\item
    \begin{tabular*}{0.97\textwidth}{l@{\extracolsep{\fill}}r}
      \textbf{#1} & #2 \\
      \textit{\small#3} & \textit{ #4} \\
    \end{tabular*}\vspace{-5pt}
}

\newcommand{\resumeSubheadingtwo}[2]{
  \vspace{-1pt}\item
    \begin{tabular*}{0.97\textwidth}{l@{\extracolsep{\fill}}r}
      \textbf{#1} & \textit{ #2} \\
      % \textit{\small#3} & \textit{\small #4} \\
    \end{tabular*}\vspace{-5pt}
}

\newcommand{\resumeSubItem}[2]{\resumeItem{#1}{#2}\vspace{-4pt}}

\renewcommand{\labelitemii}{$\circ$}

\newcommand{\resumeSubHeadingListStart}{\begin{itemize}[leftmargin=*]}
\newcommand{\resumeSubHeadingListEnd}{\end{itemize}}
\newcommand{\resumeItemListStart}{\vspace{-10pt}\begin{itemize}}
\newcommand{\resumeItemListEnd}{\end{itemize}\vspace{-10pt}}

%-------------------------------------------
%%%%%%  CV STARTS HERE  %%%%%%%%%%%%%%%%%%%%%%%%%%%%


\begin{document}

%----------HEADING-----------------
\begin{tabular*}{\textwidth}{l@{\extracolsep{\fill}}r}
  \textbf{\href{https://codeforces.com/profile/zzq229}{\Large 赵子琦}} & 邮箱 : \href{mailto:zziqi229@gmail.com}{zziqi@buaa.edu.cn}\\
   & 电话 : \href{tel:+8617813063387}{+86-173-3194-9330} \\
   & 微信号 : zzq229cr\\
\end{tabular*}

%-----------EDUCATION-----------------
\vspace{-20pt}
\section{教育经历}
  \resumeSubHeadingListStart
    \resumeSubheading
      {北京航空航天大学}{北京}
      {计算机科学与技术;  硕士在读,2025年毕业}{2022年9月 -- 至今}
    \resumeSubheading
      {哈尔滨工程大学}{哈尔滨, 黑龙江}
      {计算机科学与技术;  排名: 4/315, 前1.3\%}{2018年9月 -- 2022年6月}
  \resumeSubHeadingListEnd

\section{实习经历}
    \begin{itemize}[leftmargin=*,itemsep=-20pt]
        \resumeSubheadingtwo
            {字节跳动 Data-AML-科学计算}{2023年5月 -- 至今}
            \\[10pt]
            % \resumeItem{开发}
            负责弹性算力平台开发,基于云原生技术栈,屏蔽业务场景和资源平台的复杂性,为各类计算负载提供简洁、一致的使用接口,实现低成本、高吞吐任务计算。
            \resumeItemListStart
                 \resumeItem{}
                {基于 k8s,virtual kubelet 实现混合云调度,覆盖多个算力池,支持潮汐资源,提供统一计算服务.}
                \resumeItem{}
                {覆盖了20w 核CPU,4k块GPU,高峰同时运行 3w+ 计算容器、10w+ 计算任务,并保证高可用.}
                \resumeItem{}
                {计算任务动态分发、容器复用、多进程并发执行、失败超时重试、容器动态扩缩容、优先级等功能.}
            \resumeItemListEnd
    \end{itemize}

\section{项目经历}
    \begin{itemize}[leftmargin=*,itemsep=5pt]
        \resumeSubheadingtwo
            {MIT6.824分布式系统}{\href{https://github.com/zzq229-creator/6.824}{https://github.com/zzq229-creator/6.824}}
            % 麻省理工学院研究生课程,通过 paper reading/lectures/labs 学习和理解分布式系统的基础知识.涉及到 MapReduce、GFS、Raft、ZooKeeper、Memcached 等分布式系统领域经典论文.实验中实现MapReduce、Raft算法.
            \\[7pt]
                % {麻省理工学院研究生课程,涉及到MapReduce、GFS、Raft、ZooKeeper、Spanner、Spark等技术.并构建了一个高并发高可用的线性一致分片键值数据库.}
            \resumeItemListStart
                % \resumeItem{MapReduce}
                    % {根据MapReduce论文实现了一个分布式MapReduce框架,实现worker失效处理}
                \resumeItem{Raft算法}
                    {实现Raft,包括领导选举、日志复制、日志压缩、持久化,容忍网络丢包、网络分区、节点失效等错误.}
                \resumeItem{键值数据库}
                    {基于Raft共识构建高可用的分片键值数据库.支持增删数据节点、分片动态迁移、过期分片回收、不停机配置热更新.
                        % \begin{itemize}
                        %     \item 构建高可用的分布式分片配置管理器,支持增删数据节点。
                        %     \item 每个分片服务集群询问配置管理器负责一组分片,实现了分片迁移、垃圾分片回收、配置更改期间提供服务,并保证严格线性一致.
                        % \end{itemize}
                    }
            \resumeItemListEnd

        \resumeSubheadingtwo
            {校内信息平台}{}{}
            % {\href{https://qinglianjie.cn/}{https://qinglianjie.cn}}
            \\[7pt]
            校园内网信息一站式获取,使用微服务架构和 Celery 分布式任务队列系统。拥有 5000+ 用户,提供所有课程信息查询,课表、成绩、学分查询汇总,课程出分提醒,统计课程成绩分布,实现课程论坛. 
    \end{itemize}

\section{学术经历}
    \begin{itemize}[leftmargin=*,itemsep=-20pt]
        \resumeSubheadingtwo
            {知识图谱增强多兴趣学习的多行为推荐}{https://dl.acm.org/doi/full/10.1145/3606369,TOIS(CCF A类期刊)}
            \\[5pt]
            共同一作,首次在多行为推荐中引入多兴趣学习,挖掘用户与物品交互背后的兴趣,提出使用知识图谱提取兴趣的初始表示,结合了动态路由方案进一步探索每个行为背后的兴趣,在多个数据集上达到了最佳的推荐效果.
            % 首次在多行为推荐中引入多兴趣学习.在大多数推荐场景中存在多种行为类型(例如点击、加入购物车、购买等),不同的行为类型可能反映用户的不同兴趣。我们提出使用知识图谱提取物品的每个兴趣的初始表示,之后结合了动态路由方案进一步探索每个行为背后的兴趣。我们的模型在四个数据集上达到了最佳的推荐性能.
    \end{itemize}

\section{曾获奖项}
    \begin{itemize}[leftmargin=*,itemsep=-20pt]
        \resumeSubheadingtwo
          {2018-2019年度国家奖学金}{2018-2019年}
        \resumeSubheadingtwo
          {2019-2020年度校长奖学金}{2019-2020年}
      % \resumeSubItem{第 45 届 ICPC 国际大学生程序设计竞赛亚洲区域赛(上海站) 金牌}
        \resumeSubheadingtwo
          {第 45 届 ICPC 国际大学生程序设计竞赛亚洲区域赛\,金牌}{2020年12月}
        % \resumeSubheadingtwo
          % {第 44 届 ICPC 国际大学生程序设计竞赛亚洲区域赛(上海站) 银牌}{2019年11月}
        % \resumeSubheadingtwo
          % {第六届 CCPC 中国大学生程序设计竞赛(绵阳站)银牌}{2020年11月}
        \resumeSubheadingtwo
          {第五届 CCPC 中国大学生程序设计竞赛\,银牌}{2019年9月}
        \resumeSubheadingtwo
          {2020 CCF 中国大学生系统与程序设计竞赛全国总决赛\,金牌 (排名:28/1041)}{2020年10月}
        \resumeSubheadingtwo
          {2020 黑龙江省大学生程序设计竞赛\,亚军}{2020年9月}
        \resumeSubheadingtwo
          {2020 东北地区大学生程序设计竞赛\,一等奖(第4名)}{2020年10月}
        % \resumeSubheadingtwo
          % {2019 东北地区大学生程序设计竞赛 \, 一等奖}{2019年5月}
        \resumeSubheadingtwo
          {2019 全国大学生数学建模竞赛黑龙江赛区\,一等奖}{2019年9月}
        % \resumeSubheadingtwo
          % {第十届蓝桥杯全国软件和信息技术专业人才大赛黑龙江赛区 \, 一等奖(第4名)}{2019年3月}
    \end{itemize}

% \section{曾获荣誉}
%     \begin{itemize}[leftmargin=*,itemsep=-20pt]
%       % \resumeSubItem{第 45 届 ICPC 国际大学生程序设计竞赛亚洲区域赛(上海站) 金牌}
%         \resumeSubheadingtwo
%           {2018-2019年度国家奖学金}{2018-2019年}
%         \resumeSubheadingtwo
%           {2019-2020年度校长奖学金}{2019-2020年}
%         % \resumeSubheadingtwo
%           % {校级三好学生}{2018-2019年}
%         % \resumeSubheadingtwo
%           % {校级一等奖学金}{2018-2019,2019-2020,2020-2021}
%     \end{itemize}


%--------PROGRAMMING SKILLS------------
\section{个人}
 \resumeSubHeadingListStart
   \item{
     {热爱code \& debug,具有扎实的代码能力、优秀的算法和数据结构基础,对深度学习有一些了解}
      % {求知欲强,有比较好的优秀的学习和沟通能力}
     % \hfill
     % \textbf{Technologies}{: AWS, Play, React, Kafka, GCE}
   }
   \item{
    求知欲强,有较好的学习和沟通能力.
   }   
   % \item{
    % 能够使用Numpy、Pandas等库进行数据处理、分析
   % }
 \resumeSubHeadingListEnd

% %-----------EXPERIENCE-----------------
% \section{Experience}
%   \resumeSubHeadingListStart

%     \resumeSubheading
%       {Google}{Mountain View, CA}
%       {Software Engineer}{Oct 2016 - Present}
%       \resumeItemListStart
%         \resumeItem{Tensorflow}
%           {TensorFlow is an open source software library for numerical computation using data flow graphs; primarily used for training deep learning models.}
%         \resumeItem{Apache Beam}
%           {Apache Beam is a unified model for defining both batch and streaming data-parallel processing pipelines, as well as a set of language-specific SDKs for constructing pipelines and runners.}
%       \resumeItemListEnd

%     \resumeSubheading
%       {Coursera}{Mountain View, CA}
%       {Senior Software Engineer}{Jan 2014 - Oct 2016}
%       \resumeItemListStart
%         \resumeItem{Notifications}
%           {Service for sending email, push and in-app notifications. Involved in features such as delivery time optimization, tracking, queuing and A/B testing. Built an internal app to run batch campaigns for marketing etc.}
%         \resumeItem{Nostos}
%           {Bulk data processing and injection service from Hadoop to Cassandra and provides a thin REST layer on top for serving offline computed data online.}
%         \resumeItem{Workflows}
%           {Dataduct an open source workflow framework to create and manage data pipelines leveraging reusables patterns to expedite developer productivity.}
%         \resumeItem{Data Collection}
%           {Designed the internal survey and crowd sourcing platfowm which allowed for creating various tasks for crowd sourding or embedding surveys across the Coursera platform.}
%         \resumeItem{Dev Environment}
%           {Analytics environment based on docker and AWS, standardized the python and R dependencies. Wrote the core libraries that are shared by all data scientists.}
%         \resumeItem{Data Warehousing}
%           {Setup, schema design and management of Amazon Redshift. Built an internal app for access to the data using a web interface. Dataduct integration for daily ETL injection into Redshift.}
%         \resumeItem{Recommendations}
%           {Core service for all recommendation systems at Coursesa, currently used on the homepage and throughout the content discovery process. Worked on both offline training and online serving.}
%         \resumeItem{Content Discovery}
%           {Improved content discovery by building a new onboarding experience on coursera. Using this to personalize the search and browse experience. Also worked on ranking and indexing improvements.}
%         \resumeItem{Course Dashboards}
%           {Instructor dashboards and learner surveying tools, which helped instructors run their class better by providing data on Assignments and Learner Activity.}
%       \resumeItemListEnd

%     \resumeSubheading
%       {Lucena Research}{Atlanta, GA}
%       {Data Scientist}{Summer 2012 and 2013}
%       \resumeItemListStart
%         \resumeItem{Portfolio Management}
%           {Created models for portfolio hedging,  portfolio optimization and price forecasting. Also creating a strategy backtesting engine used for simulating and backtesting strategies.}
%         \resumeItem{QuantDesk}
%           {Python backend for a web application used by hedge fund managers for portfolio management.}
%       \resumeItemListEnd

%   \resumeSubHeadingListEnd


%-----------PROJECTS-----------------
% \section{Projects}
%   \resumeSubHeadingListStart
%     \resumeSubItem{QuantSoftware Toolkit}
%       {Open source python library for financial data analysis and machine learning for finance.}
%     \resumeSubItem{Github Visualization}
%       {Data Visualization of Git Log data using D3 to analyze project trends over time.}
%     \resumeSubItem{Recommendation System}
%       {Music and Movie recommender systems using collaborative filtering on public datasets.}
% %     \resumeSubItem{Mac Setup}
% %       {Book that gives step by step instructions on setting up developer environment on Mac OS.}
%   \resumeSubHeadingListEnd

% %
%--------PROGRAMMING SKILLS------------
% \section{编程语言}
%  \resumeSubHeadingListStart
%    \item{
%      \textbf{}{C/C++, Python, Golang, Java}
%      % \hfill
%      % \textbf{Technologies}{: AWS, Play, React, Kafka, GCE}
%    }
%  \resumeSubHeadingListEnd


%-------------------------------------------
\end{document}
